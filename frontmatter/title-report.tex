\begin{titlepage}

\begin{center}

%% Print the title
{\makeatletter
\largetitlestyle\fontsize{45}{45}\selectfont\@title
\makeatother}

%% Print the subtitle
{\makeatletter
\ifdefvoid{\@subtitle}{}{\bigskip\titlestyle\fontsize{16}{14}\selectfont\@subtitle}
\makeatother}

\bigskip
\bigskip

%by

\bigskip
\bigskip

%% Print the name of the author
{\makeatletter
\largetitlestyle\fontsize{25}{25}\selectfont\@author
\makeatother}

\bigskip
\bigskip

%% Print table with names and student numbers
%\setlength\extrarowheight{2pt}
%\begin{tabular}{lc}
%
%Student Name & Student Number \\\midrule
%
%First Surname & 123456 \\
%\end{tabular}

\vfill

%% Print some more information at the bottom
% Título de la serie
% Título y Descripción General
\vspace*{\fill} % Centra el contenido verticalmente
\begin{center}
    {\Huge \textbf{Ética en IA: Una Serie de la Comunidad Python Colombia y Python Barranquilla}} \\[15pt]
    {\Large \textbf{Parte I: Interpretabilidad y Explicabilidad en IA}} \\[30pt]
    
    {\large Este documento forma parte de la serie \textbf{Ética en IA}, desarrollada en colaboración por las comunidades \textbf{Python Colombia} y \textbf{Python Barranquilla}.} \\[15pt]
    
    {\large La serie está estructurada en tres partes, cada una abordando un tema fundamental para el desarrollo ético y responsable de la IA:}
    \begin{enumerate}[label=\arabic*.]
        \item \textbf{Parte I: Interpretabilidad y Explicabilidad en IA:} Fundamentos, técnicas, estado del arte y aplicaciones prácticas para construir sistemas de IA comprensibles.
        \item \textbf{Parte II: Justicia Algorítmica:} Identificación y mitigación de sesgos algorítmicos en sistemas inteligentes.
        \item \textbf{Parte III: Ética para la Protección del Futuro de los Recursos Genéticos:} Exploración crítica y multidisciplinaria de los desafíos éticos, legales y técnicos relacionados con el manejo de recursos genéticos y secuencias digitales. Este apartado abarca:
        \begin{itemize}
            \item \textit{Ciencia Abierta:} Oportunidades y riesgos asociados a la accesibilidad de datos genéticos y su impacto en la innovación científica.
            \item \textit{CRISPR y Tecnologías de Edición Genética:} Implicaciones éticas y regulatorias del uso de herramientas avanzadas en la manipulación genética.
            \item \textit{Secuencias Genéticas Digitales (DSI):} Consideraciones sobre el almacenamiento, uso y comercio de información genética en formato digital.
            \item \textit{Prevención de la Biopiratería:} Estrategias para garantizar la equidad y el respeto por los recursos genéticos, especialmente en comunidades vulnerables.
        \end{itemize}

    \end{enumerate}
    
    {\large En esta primera parte, exploraremos la \textbf{Interpretabilidad y Explicabilidad} de los sistemas de IA, combinando teoría, práctica y reflexión ética.} \\[25pt]
    
    {\large \textbf{Duración de la Primera Parte:} Desde el martes 3 de diciembre de 2024 hasta el martes 25 de febrero de 2025. Las sesiones se llevarán a cabo todos los martes de 7:00 PM a 9:00 PM (excepto el 24 y 31 de diciembre de 2024).} \\[15pt]
    
    {\large \textit{Aprenderemos cómo diseñar y evaluar sistemas de IA que no solo sean precisos, sino también comprensibles para los usuarios y éticamente responsables.}}
\end{center}
\vspace*{\fill}

% Información adicional en la parte inferior
\noindent
\rule{\textwidth}{0.4pt} % Línea divisoria
\vspace{5pt}
\begin{tabular}{lp{12cm}}
    \textbf{Grupo de Estudio:} & Serie Ética en IA (Parte I: Interpretabilidad y Explicabilidad). \\[5pt]
    \textbf{Duración:} & Desde el 3 de diciembre de 2024 hasta el 25 de febrero de 2025 (excepto el 24 y 31 de diciembre de 2024). \\[5pt]
    \textbf{Horario:} & Todos los martes de 7:00 PM a 9:00 PM. \\[5pt]
    \textbf{Organizado por:} & Comunidad Python Colombia y Python Barranquilla. \\[5pt]
%    \textbf{Coordinadores:} & [Espacio para agregar nombres]. \\[5pt]
%    \textbf{Contacto:} & [Espacio para agregar información de contacto o redes sociales]. \\
\end{tabular}


\bigskip
\bigskip

%% Add a source and description for the cover and optional attribution for the template
%\begin{tabular}{p{15mm}p{10cm}}
%    Cover: & The Rolex Learning Center at EPFL (Modified) \\
%     Feel free to remove the following attribution, it is not required - still appreciated :-)
%    Style: & EPFL Report Style, with modifications by Batuhan Faik Derinbay
%\end{tabular}

\end{center}

%% Insert the EPFL logo at the bottom of the page
\begin{tikzpicture}[remember picture, overlay]
    \node[above=10mm] at (current page.south) {%
         %\includegraphics[width=0.35\linewidth]{layout/epfl/logo-black}
         \includegraphics[width=0.35\linewidth]{layout/epfl/cc}
    };
\end{tikzpicture}

\end{titlepage}
