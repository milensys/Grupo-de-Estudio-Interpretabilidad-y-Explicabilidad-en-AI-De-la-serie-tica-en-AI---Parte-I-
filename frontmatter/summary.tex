\chapter*{Resumen de la Primera Parte: Interpretabilidad y Explicabilidad en IA}
\addcontentsline{toc}{chapter}{Resumen de la Parte I: Interpretabilidad y Explicabilidad en IA}

La primera parte del grupo de estudio sobre ética en IA se centra en uno de los pilares fundamentales para el desarrollo ético y responsable de sistemas basados en IA: la \textbf{interpretabilidad y explicabilidad}. En un contexto donde los sistemas de IA son cada vez más complejos y se aplican en áreas críticas, garantizar que sus decisiones sean comprensibles es clave para fomentar la transparencia, confianza y adopción adecuada.

\section*{Objetivos Clave}
\begin{itemize}
    \item Proveer una base sólida en conceptos, técnicas, estado del arte y herramientas modernas de interpretabilidad y explicabilidad.
    \item Capacitar a los participantes en la implementación práctica de métodos para analizar modelos de caja negra.
    \item Explorar aplicaciones de explicabilidad en sectores clave como salud, finanzas y educación.
    \item Reflexionar sobre los desafíos éticos y sociales relacionados con la transparencia y la confianza en la IA.
\end{itemize}

\section*{Requisitos para los Participantes}
Cada participante deberá leer de manera anticipada el material de estudio asignado para cada sesión, con el fin de maximizar su comprensión y participación activa. Este material incluirá capítulos de libros, artículos académicos y recursos técnicos.

\section*{Contenidos Destacados}
\begin{enumerate}
    \item \textbf{Fundamentos de Interpretabilidad y Explicabilidad:}
    \begin{itemize}
        \item Diferencias clave entre interpretabilidad (\textit{comprensión intrínseca}) y explicabilidad (\textit{mecanismos post-hoc}).
        \item Importancia de estas características en sistemas críticos y regulados.
    \end{itemize}
\newpage
    \item \textbf{Técnicas Clásicas:}
    \begin{itemize}
        \item \textbf{SHAP} y \textbf{LIME}: Métodos ampliamente utilizados para explicar modelos de aprendizaje automático.
        \item \textbf{Eli5} y \textbf{Surrogate Models}: Herramientas para analizar decisiones en modelos complejos.
    \end{itemize}

    \item \textbf{Técnicas Visuales:}
    \begin{itemize}
        \item \textbf{Grad-CAM} y \textbf{Saliency Maps}: Métodos para interpretar redes neuronales aplicadas a visión por computadora.
        \item Interpretación de modelos basados en \textbf{mecanismos de atención}, como transformers.
    \end{itemize}

    \item \textbf{Técnicas Emergentes:}
    \begin{itemize}
        \item \textbf{Contrafactuales:} Generación de escenarios hipotéticos para comprender decisiones de modelos de IA.
        \item Explicabilidad en transformers y modelos generativos, incluyendo aplicaciones multimodales.
    \end{itemize}

    \item \textbf{Evaluación de Explicaciones:}
    \begin{itemize}
        \item Métodos y métricas para medir la calidad, robustez y utilidad de las explicaciones generadas.
    \end{itemize}
\end{enumerate}

\section*{Metodología}
\begin{itemize}
    \item \textbf{Enfoque práctico:} Implementación de técnicas de explicabilidad en Python, utilizando bibliotecas como \texttt{Scikit-learn}, \texttt{SHAP}, y \texttt{TensorFlow}.
    \item \textbf{Reflexión ética:} Discusiones en cada sesión sobre las implicaciones sociales y éticas de la explicabilidad en sistemas de IA.
    \item \textbf{Aplicaciones reales:} Trabajo con \textbf{datasets representativos} en áreas como predicción de enfermedades, análisis financiero y recomendaciones educativas.
\end{itemize}

\section*{Algunos Datasets Representativos}
\begin{itemize}
    \item \textbf{Iris Dataset:} Un clásico para tareas de clasificación multiclase.
        \begin{itemize}
            \item URL: \url{https://archive.ics.uci.edu/ml/datasets/iris}
        \end{itemize}
    \item \textbf{Titanic Dataset:} Análisis de supervivencia basado en características personales.
        \begin{itemize}
            \item URL: \url{https://www.kaggle.com/c/titanic/data}
        \end{itemize}
    \item \textbf{MNIST Dataset:} Reconocimiento de dígitos manuscritos, ideal para técnicas visuales.
        \begin{itemize}
            \item URL: \url{https://www.kaggle.com/c/digit-recognizer/data}
        \end{itemize}
    \item \textbf{German Credit Dataset:} Evaluación de solicitudes de crédito para análisis de riesgo financiero.
        \begin{itemize}
            \item URL: \url{https://archive.ics.uci.edu/ml/datasets/statlog+(german+credit+data)}
        \end{itemize}
    \item \textbf{IMDB Reviews Dataset:} Análisis de sentimientos en texto.
        \begin{itemize}
            \item URL: \url{https://ai.stanford.edu/~amaas/data/sentiment/}
        \end{itemize}
\end{itemize}

\section*{Beneficios para los Participantes}
Al finalizar esta primera parte, los asistentes adquirirán:
\begin{itemize}
    \item Habilidades prácticas para explicar modelos complejos y comunicar resultados a audiencias técnicas y no técnicas.
    \item Dominio de herramientas modernas y conocimiento de técnicas emergentes en explicabilidad.
    \item Conciencia crítica sobre los desafíos éticos y sociales de la IA explicable.
\end{itemize}

\section*{Espacios para Moderadores y Expositores}
Cada sesión contará con un moderador y un expositor designado para coordinar las discusiones y actividades prácticas. Estos roles se asignarán con antelación para garantizar una preparación adecuada de cada sesión con antelación.
